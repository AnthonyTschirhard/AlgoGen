%---Packages---%
\documentclass[a4paper,11pt]{article}
\usepackage[left=2.5cm,top=2cm,right=2cm,nohead]{geometry}
\usepackage[french]{babel}
\usepackage[T1]{fontenc}
\usepackage[utf8]{inputenc} 
\usepackage{graphicx}
\usepackage{float}
\usepackage{amsmath}
\usepackage{amsfonts}
\usepackage{amssymb}
\usepackage{listings}
\usepackage{mdwlist}
\usepackage[usenames,dvipsnames]{color}
\usepackage[stable]{footmisc}%To include footnotes in 'section' parts
\usepackage{hyperref}
\usepackage{setspace}
\usepackage{eurosym}
\usepackage[section]{algorithm} % [section] is use to define the numbering mode
\usepackage{algorithmic} 

%---Insertion de code---%
\definecolor{lightgray}{gray}{0.95}


\lstset
{           
backgroundcolor=\color{lightgray},
keywordstyle=\color{Red}\bfseries,
ndkeywordstyle=\color{darkgray}\bfseries,
commentstyle=\color{Green},
stringstyle=\color{Orange},
basicstyle=\footnotesize,       % the size of the fonts that are used for the code
numbers=left,                   % where to put the line-numbers
numberstyle=\footnotesize,      % the size of the fonts that are used for the line-numbers
stepnumber=2,                   % the step between two line-numbers. If it's 1 each line will be numbered 
numbersep=5pt,                  % how far the line-numbers are from the code
showspaces=false,               % show spaces adding particular underscores
showstringspaces=false,         % underline spaces within strings
showtabs=false,                 % show tabs within strings adding particular underscores
tabsize=2,	                % sets default tabsize to 2 spaces
captionpos=b,                   % sets the caption-position to bottom
breaklines=true,                % sets automatic line breaking
breakatwhitespace=false,        % sets if automatic breaks should only happen at whitespace
title=\lstname,                 % show the filename of files included with \lstinputlisting & 
escapeinside={\%*}{*)},         % if you want to add a comment within your code
morekeywords={*,...}            % if you want to add more keywords to the set
extendedchars=true
}

%---Liens---%
\hypersetup{
unicode=false,          % non-Latin characters in Acrobat’s bookmarks
pdftoolbar=true,        % show Acrobat’s toolbar?
pdfmenubar=true,        % show Acrobat’s menu?
pdffitwindow=false,     % window fit to page when opened
pdfstartview={FitH},    % fits the width of the page to the window
pdftitle={Projet AGGP - Synthèse des articles},    % title
pdfauthor={Balthazar Rouberol, Anthony Tschirhard, Marion Poirel, Marie Paturel},     % author
pdfsubject={Projet AGGP - Dossier d'init},   % subject of the document
pdfcreator={Balthazar Rouberol, Anthony Tschirhard, Marion Poirel, Marie Paturel},   % creator of the document
pdfkeywords={Réseaux biologiques, Réseaux, AlgoGen}, % list of keywords
pdfnewwindow=true,      % links in new window
colorlinks=true,       % false: boxed links; true: colored links
linkcolor=black,          % color of internal links
citecolor=black,        % color of links to bibliography
filecolor=white,      % color of file links
urlcolor= NavyBlue,           % color of external links
bookmarks=true,% show bookmarks bar?
bookmarksopen=false,
bookmarksnumbered = false      
}%



\begin{document}
\maketitle

\section{Introduction}
%Marie
L'objectif de ce document est de prévoir le déroulement futur du projet sur un maximum d'aspects, et de clarifier les éléments touchant à la gestion du projet. Ainsi, sa lecture devrait permettre de connaître précisément l'organisation du projet, à la fois d'un point de vue humain (répartition de l'équipe en sous-équipes etc.) et d'un point de vue des livrables à produire, ou des tâches à effectuer (estimation et répartition des tâches etc.).
%Pfiou c'était dur

\section{Contexte et rappel du problème}
\subsection{Contexte}
%Balto
L'étude des systèmes complexes par modélisation sous forme de réseau est un moyen efficace de comprendre leur fonctionnement intrinsèque. En effet, il est possible de relier la structure du système aux fonctions de ses composantes. A l'heure actuelle, les réseaux biologiques sont très étudiés : 

%l'émergence de moyens de communication est à l'origine d'une organisation originale qu'il peut s'avérer très utile de comprendre. Par exemple, la structure d'un réseau social permet de comprendre comment l'information se diffuse ainsi que le rôle tenu par les utilisateurs dans son relai.

Les réseaux sociaux sont caractérisés par une architecture \og sans-échelle \fg {}, dans laquelle certaines composantes jouent un rôle plus important que d'autres : les hubs. On citera les exemples de Google, Facebook et Twitter sur le réseau Internet ou encore de personnalités éminentes. Cette architecture implique une distribution de degrés des n\oe uds selon une loi de puissance de paramètre $\gamma$, compris entre 2 et 3, comme pour la grande majorité des réseaux biologiques ($\gamma_{moy}\simeq 2.1$). Une autre caractéristique d'un tel réseau est la formation de cliques au sein de sa structure.

Savoir comment un tel réseau évolue au cours du temps est aujourd'hui une préoccupation grandissante car cela permet de mieux les maîtriser. Un autre point intéressant est la recherche de conditions qui amènent un réseau à avoir la structure particulière qui lui est propre.

\subsection{Objectifs}
%Balto
Le but de ce projet est de générer un réseau social en utilisant un algorithme génétique, développé en séance d'Optimisation. Il faudra pour cela décider d'une fonction de fitness adaptée, afin que les réseaux générés se rapprochent de génération en génération d'un réseau social \og idéal \fg .

Les propriétés d'architecture \og sans-échelle \fg{} et de modularité du réseau sont relativement contradictoires : la présence de hubs rend improbable l'isolation de certains n\oe{}uds du réseau, ce qui est pourtant suggéré par la formation de cliques. Il faudra donc construire la fonction de fitness autour d'un compromis acceptable entre ces deux conditions.

\section{Documents de référence}
%Balto, Marion et Anthony
Les connaissances générales sur la structure, l'architecture et le comportement des réseaux biologiques nous viennent de deux articles :

\begin{itemize}
	\item "Exploring complex networks", \textit{Steven H. Strogatz}, Nature, Vol. 410, March 2001
	\item "Network biology : understanding the cell's functional organization", \textit{Albert-L\'{a}zlo Barab\'{a}si, Zolt\'{a}n N. Oltvai}, Nature reviews, Genetics, Vol. 5, February 2001\medskip
\end{itemize}

%A compléter après ajout de biblio, par Anthony et Marion

L'algorithme génétique est celui contenu dans le fichier correction \texttt{pyAG.py}, fourni par M. Hédi Soula.

\section{Contraintes générales}

\subsection{Contraintes}
% Marion
Tout projet est gouverné par la triade des contraintes de coût, délai, qualité. Ici, celle de coût est négligeable, s'agissant simplement d'un projet scolaire. Il convient cependant de trouver un équilibre entre la contrainte de temps et celle de qualité.

L'aspect temporel est ici primordial : ce projet présente une contrainte calendaire imposée qui ne peut être contournée (dates de rendu des différents livrables et de la soutenance). Le délai étant relativement court, il est important de ne pas se faire prendre par le temps.

La qualité est cruciale. Elle sera garante de résultats fiables, exploitables et reproductibles. 

L'échéance étant proche, il est important d'établir un compromis entre temps et qualité. Prévoir une marge de manœuvre autorisera une bonne conduite du projet.\\

Une exigence est l'utilisation du package \verb?NetworkX? de Python. Il permet la création et l'exploitation de graphes complexes de type réseau. Le maîtriser est une étape nécessaire, préalable à celle de programmation.

Une exploitation du code est attendue. Une des finalités de l'exercice étant de faire émerger des propriétés connues de réseaux biologiques et d'étudier leur apparition.

Enfin, l'exigence principale est l'apprentissage de la gestion d'un projet : travail en groupe, répartition des tâches, dossiers à rendre, etc. sont autant d'aspects dont nous devons tenir compte.

\subsection{Risques}
% Marion
Les risques à considérer sont de deux sortes : risques liés au sujet d'étude et risques liés à la gestion de projet.\\

Rapport au sujet d'étude, il est à craindre une mauvaise compréhension conduisant à une approche non adaptée. Cela pourrait conduire à omettre une problématique importante et donc à réaliser une étude incomplète, voire incorrecte.\\

Quant à la gestion de projet en elle-même, elle peut faire émerger des difficultés aussi bien organisationnelles que relationnelles ou encore matérielles.

L'absence ou le manque de travail d'un des membres du groupe peut aboutir à un retard pouvant mettre en péril l'aboutissement du projet. Le chef de projet se doit de veiller à la motivation et l'implication de l'équipe afin de prévenir de tels risques ainsi qu'un manque de productivité, notamment lors des réunions. L'ambiance de groupe a également son importance, influençant directement la communication et la motivation.

Par ailleurs, une mauvaise gestion du temps peut mettre à mal le projet. L'emploi du temps de chacun étant chargé, il faut faire attention à en respecter les contraintes.

Enfin, les problèmes matériels peuvent fortement influencer l'avancement du projet. Un mauvais fonctionnement des machines et/ou du réseau peut bloquer le travail. La perte de données peut avoir des conséquences dramatiques, aussi bien sur le moral que sur l'avancement général.

\section{Organisation du travail}
\subsection{Rôles distribués}

\subsubsection{Chef de Projet}
\textbf{Balthazar Rouberol}
Le chef de projet a un rôle de coordinateur : il est responsable du bon déroulement du projet. Il doit décomposer le projet en différentes tâches et répartir les tâches aux membres du groupe. Il constitue le planning prévisionnel et supervise le travail de chacun jusqu'à la présentation du projet.
  
\subsubsection{Responsable Qualité}
\textbf{Marion Poirel}
Elle contrôle la qualité des biens produits par le groupe. Elle doit définir les règles de qualité et donner des recommandations aux autres membres. Elle effectue la relecture des documents.

  
\subsubsection{Responsable Qualité}
\textbf{Marion Poirel}\\
Elle contrôle la qualité des documents produits par le groupe. Elle doit définir les règles de qualité et donner des recommandations aux autres membres du groupe. Elle effectue la relecture des documents produits.

\subsubsection{Responsable bibliographie}
\textbf{Anthony Tschirhard}
Il recherche un maximum d'informations sur le sujet des réseaux, et notamment des réseaux sociaux, afin de compléter les documents de référence donnés par le professeur.

\subsubsection{Responsable Documentation}
\textbf{Marie Paturel}
Elle est en charge de la compréhension de la documentation, et notamment du package \verb?NetworkX?. Elle doit résumer aux autres membres le fonctionnement de ce dernier afin qu'il soit utilisable par l'ensemble du groupe.

\subsubsection{Groupe d'étude informatique}
\textbf{Balthazar Rouberol, Anthony Tschirhard, Marion Poirel, Marie Paturel}

Le groupe d'étude informatique gère l'analyse du sujet et la rédaction des documents du projet. Ce groupe suit les tâches affectées par le chef de projet et les recommandations du responsable qualité. Il rédige des documents tels que le dossier d'initialisation et le dossier de conception.

Le groupe d'étude informatique a également pour charge de développer le programme répondant aux attentes du projet, ainsi que de le débugger.


\textbf{Marie Paturel}\\
Elle est en charge de la compréhension de la documentation, et notamment du package \texttt{NetworkX}. Elle doit résumer aux autres membres le fonctionnement de ce dernier afin qu'il soit utilisable par l'ensemble du groupe.

\subsubsection{Groupe d'étude informatique}
\textbf{Balthazar Rouberol, Anthony Tschirhard, Marion Poirel, Marie Paturel}\\
Le groupe d'étude informatique gère l'analyse du sujet et la rédaction des documents du projet. Ce groupe suit les tâches affectées par le chef de projet et suit les recommandations du responsable qualité. Il rédige des documents tels que le dossier d'initialisation et le dossier de conception.\\
Le groupe d'étude informatique a également pour charge de développer le programme répondant aux attentes du projet, ainsi que de le débugguer.


\subsection{Règles de suivi}
%Anthony
Le suivi des tâches et des implications de chacun se fera en continu tout au long de ce projet. Au début de chaque session de travail, l'équipe se réunira afin de :
\begin{itemize}
  \item contrôler l'avancement du travail de chacun et les éventuels problèmes rencontrés ;
  \item prendre des mesures correctives en cas de retard dans le projet ;
  \item faire le bilan sur la partie du projet actuellement abordée ;
  \item prendre des mesures correctives en cas de retard dans le projet ;
  \item répartir le travail pour la séance.
\end{itemize}

\medskip
En fin de chaque session de travail, l'équipe se réunira une nouvelle fois afin de :
\begin{itemize}
  \item contrôler l'avancement du travail de chacun et les éventuels problèmes rencontrés ;
  \item faire de bilan de la séance achevée ;
  \item répartir le travail pour la prochaine session ;
  \item définir une date pour la prochaine séance de travail.
\end{itemize}

Les réunions auront lieu deux fois par semaine : le mercredi, durant les heures dédiées dans l'emploi du temps, et le vendredi. Ceci permet de suivre l'évolution du projet en continu et de réagir rapidement en cas de problème.


\subsection{Organigramme des tâches}
%Anthony
\paragraph*{Compréhension du corpus bibliographique réalisé\\}
\textbf{Analyse des besoins :} 6~heures par personne\\
\textbf{Date limite fixée pour le rendu :} 17/03/2011\\
\textbf{Objectif : }s'approprier le sujet que l'on souhaite traiter afin de répondre aux objectifs\\
\textbf{Rôle du responsable documentation :} synthétiser les différentes informations recueillies

\paragraph*{Compréhension du module \verb?NetworkX?\\}
\textbf{Analyse des besoins :} 10~heures pour 2~personnes\\
\textbf{Date limite fixée pour le rendu :} 20/03/2011\\
\textbf{Objectif :} étude du module \verb?NetworkX? : réalisation d'une documentation et d'un tutoriel pour rendre son utilisation familière aux autres membres du groupe\\
\textbf{Rôle du maitre d'ouvrage : }étudier la stratégie de réalisation de la documentation et du tutoriel\\
\textbf{Rôle du responsable documentation : }réaliser un tutoriel et une documentation pour le module

\paragraph*{Modélisation de notre fonction de \textit{fitness}\\}
\textbf{Analyse des besoins : }6~heures\\
\textbf{Date limite fixée pour le rendu :} 20/03/2011\\
\textbf{Objectif : }le point crucial du projet est de définir de manière la plus juste possible une fonction de fitness pour le modèle de réseaux réalisé. Cette fonction doit être adaptée à nos objectifs et correspondre à la réalité observée\\
\textbf{Rôle du maitre d'ouvrage : }analyser les besoins et les contraintes du système du réseau choisi et réfléchir sur l'implémentation de la fonction de fitness\\
\textbf{Rôle du responsable documentation : }vérifier que le modèle est bien en accord avec la bibliographie réalisée

\paragraph*{Codage du réseau et sorties graphiques\\}
\textbf{Analyse des besoins : }2~semaines\\
\textbf{Date limite fixée pour le rendu : }04/04/2011\\
\textbf{Objectif : }implémenter le programme générant des réseaux biologiques par la technique d'algorithme génétique\\
\textbf{Rôle du maitre d'ouvrage : }définir l'organisation du code et ses normes syntaxiques. Répartir avec le chef de projet le développement des différentes sous-parties du programme\\
\textbf{Rôle du responsable qualité : }relire le code et les commentaires\\
\textbf{Rôle du responsable documentation : }vérifier la cohérence du code avec les articles de la bibliographie

\paragraph*{Phase de débogage du code\\}
\textbf{Analyse des besoins :} 10~heures pour 2~personnes\\
\textbf{Date limite fixée pour le rendu :} 08/04/2011\\
\textbf{Rôle du maitre d'ouvrage :} planifier les différentes phases de débogage et les méthodes pour y arriver\\
\textbf{Rôle du responsable qualité :} vérifier la bonne marche du programme

\paragraph*{Réalisation de la présentation}
\textbf{Analyse des besoins : }4~heures\\
\textbf{Date limite fixée pour le rendu : }10/04/2011\\
\textbf{Objectif : }réaliser une présentation synthétique des résultats obtenus\\
\textbf{Rôle du responsable qualité : } examiner le contenu de la présentation


\subsection{Outils utilisés}
%Anthony
Afin de gérer au mieux ce projet, il convenait de mettre toutes les chances de notre côté en utilisant des outils adaptés à ce genre de travaux. Deux outils nous paraissaient particulièrement nécessaires :
\begin{itemize}
  \item un outil permettant de partager du code, à la manière de \textit{subversion}, afin que nous puissions tous travailler en collaboration de manière efficace ;
  \item un outil de gestion de projet, afin que nous puissions tous définir précisément nos objectifs, nos contraintes temporelles et avancées.
\end{itemize}

\paragraph*{Git\\}

Pour le logiciel de gestion de versions, nous avons décidé d'utiliser \textbf{Git}, un logiciel libre simple et efficace, dont la principale tâche est de gérer l'évolution du contenu d'une arborescence.

%Un logiciel de gestion de versions agit sur une arborescence de fichiers afin de conserver toutes les versions des fichiers, ainsi que les différences entre les fichiers.

Ce système nous permettra de mutualiser le développement de notre projet. Nous nous servirons de cet outil pour stocker toute évolution des codes sources ou des rapports -- le système conserve en effet une trace de chaque changement. Chacun doit être accompagné d'un commentaire. Travaillant par fusion de copies locale et distante, nous pourrons travailler de concert sur une même source, les changements de l'un n'affectant pas ceux de l'autre.

Pour héberger ce projet, nous avons décidé d'utiliser le site de stockage de projet \textbf{GitHub} qui propose notamment des fonctionnalités de type réseaux sociaux -- suivi de personnes ou de projets, graphes de réseau pour les dépôts, etc.

Il a été convenu la chose suivante : commencer par récupérer le dépôt et le fusionner avant de commencer une nouvelle session de travail, effectuer des \textit{commit} réguliers et bien commentés, et enfin pousser les modifications en fin de session.

\paragraph*{Redmine\\}

Afin de nous organiser et de mettre en forme les différents défis auxquels nous allons devoir faire face, nous avons décidé d'utiliser une application web Open Source de gestion de projet en mode web : \textbf{Redmine}. Ses principales fonctionnalités sont la gestion multi-projets, la gestion des développeurs et de leurs rôles, des notifications par mail, des historiques, forums, wiki etc.

Nous allons utiliser cette plateforme pour nous organiser dans notre travail -- collectif et individuel :

\begin{itemize}
  \item visualisation des différentes tâches à accomplir et les délais impartis ;
  \item consultation des demandes en cours et émission de nouvelles (en précisant l'échéance, la durée, le statut, à qui elle va être assignée etc.) ;
  \item évaluation de l'avancée de chacun.
\end{itemize}


\section{Conclusion}
%Marie
Ce dossier présente la gestion d'un projet à réaliser dans un temps assez restreint. Il faudra donc tâcher de travailler le plus efficacement pendant les séances prévues à cet effet, afin d'éviter de se laisser déborder et d'accumuler du travail hors séances.

Les possibilités de retard étant nombreuses, le rôle de chef de projet sera primordial dans la définition des tâches et du planning associé.

L'implication des membres du groupe doit être totale afin de respecter les dates de rendu des documents ainsi que les règles de qualité.

\end{document}