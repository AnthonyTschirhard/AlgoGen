%---Packages---%
\documentclass[a4paper,11pt]{article}
\usepackage[left=2.5cm,top=2cm,right=2cm,nohead]{geometry}
\usepackage[french]{babel}
\usepackage[T1]{fontenc}
\usepackage[utf8]{inputenc} 
\usepackage{graphicx}
\usepackage{float}
\usepackage{amsmath}
\usepackage{amsfonts}
\usepackage{amssymb}
\usepackage{listings}
\usepackage{mdwlist}
\usepackage[usenames,dvipsnames]{color}
\usepackage[stable]{footmisc}%To include footnotes in 'section' parts
\usepackage{hyperref}
\usepackage{setspace}
\usepackage{eurosym}
\usepackage[section]{algorithm} % [section] is use to define the numbering mode
\usepackage{algorithmic} 

%---Insertion de code---%
\definecolor{lightgray}{gray}{0.95}


\lstset
{           
backgroundcolor=\color{lightgray},
keywordstyle=\color{Red}\bfseries,
ndkeywordstyle=\color{darkgray}\bfseries,
commentstyle=\color{Green},
stringstyle=\color{Orange},
basicstyle=\footnotesize,       % the size of the fonts that are used for the code
numbers=left,                   % where to put the line-numbers
numberstyle=\footnotesize,      % the size of the fonts that are used for the line-numbers
stepnumber=2,                   % the step between two line-numbers. If it's 1 each line will be numbered 
numbersep=5pt,                  % how far the line-numbers are from the code
showspaces=false,               % show spaces adding particular underscores
showstringspaces=false,         % underline spaces within strings
showtabs=false,                 % show tabs within strings adding particular underscores
tabsize=2,	                % sets default tabsize to 2 spaces
captionpos=b,                   % sets the caption-position to bottom
breaklines=true,                % sets automatic line breaking
breakatwhitespace=false,        % sets if automatic breaks should only happen at whitespace
title=\lstname,                 % show the filename of files included with \lstinputlisting & 
escapeinside={\%*}{*)},         % if you want to add a comment within your code
morekeywords={*,...}            % if you want to add more keywords to the set
extendedchars=true
}

%---Liens---%
\hypersetup{
unicode=false,          % non-Latin characters in Acrobat’s bookmarks
pdftoolbar=true,        % show Acrobat’s toolbar?
pdfmenubar=true,        % show Acrobat’s menu?
pdffitwindow=false,     % window fit to page when opened
pdfstartview={FitH},    % fits the width of the page to the window
pdftitle={Projet AGGP - Synthèse des articles},    % title
pdfauthor={Balthazar Rouberol, Anthony Tschirhard, Marion Poirel, Marie Paturel},     % author
pdfsubject={Projet AGGP - Dossier d'init},   % subject of the document
pdfcreator={Balthazar Rouberol, Anthony Tschirhard, Marion Poirel, Marie Paturel},   % creator of the document
pdfkeywords={Réseaux biologiques, Réseaux, AlgoGen}, % list of keywords
pdfnewwindow=true,      % links in new window
colorlinks=true,       % false: boxed links; true: colored links
linkcolor=black,          % color of internal links
citecolor=black,        % color of links to bibliography
filecolor=white,      % color of file links
urlcolor= NavyBlue,           % color of external links
bookmarks=true,% show bookmarks bar?
bookmarksopen=false,
bookmarksnumbered = false      
}%



\begin{document}
\maketitle

%%%%%%%%%%%%%%%%%%%%%%%%%%%%%%%%%%%%%%%%%
\section{Présentation générale du projet}
% Balto


%%%%%%%%%%%%%%%%%%%%%%%
\section{Les objectifs}
% Marion
\subsection*{Algorithme génétique}

\subsection*{Gestion de projet}

\subsection*{Exploitation des résultats}


%%%%%%%%%%%%%%%%%%%%%%%%%%%%%%%%%%%%%%%%%%%%%%%%%%%%%
\section{Les grandes propriétés du réseau biologique}
% Anthony
Afin d'être le plus proche possible d'un réseau biologie, il convient de connaitre ses caractéristiques et paramètres associés. Les réseaux que nous allons créer et faire évoluer vont devoir répondre à trois grandes caractéristiques : la répartition de ses degrés devra suivre une \textit{loi exponentielle}, il devra vérifier la propriété dite de \textit{\og petit monde \fg} et devra contenir des \textit{cliques}.

\subsection{Répartition des degrés -- Loi exponentielle}
% Anthony
\begin{figure}[!h]
\includegraphics[width=\linewidth]{plot.png}
\caption{Réseau dont la distribution suit une loi exponentielle}
\label{scalefree}
\end{figure}
La répartition des degrés d'un réseau dit \textit{scale-free} suit une loi exponentielle. Comme le montre la figure~\ref{scalefree}, les degrés des nœuds d'un réseau diminuent quand leur nombre augmente. Autrement dit, on va avoir très peu de nœuds très fortement connectés et beaucoup de nœuds fortement connectés. La loi que suit cette répartition de degrés est la suivante :
$$ P(k) ~ \alpha k^{-g} $$
On a donc des nœuds très connectés appelés \textbf{hubs} et une très grande majorité de nœuds très faiblement connectés.	Les articles étudiés dans le cadre de ce projet nous informent que $g$ doit être compris entre 

%-----------------------------------
%Un reseau scale-free possede une repartition des degres en loi de puissance (P(k) 
%k􀀀g) c'est-a-dire qu'il possede des noeuds tres connectes appeles hubs et une ma-
%jorite de noeuds qui sont faiblement connectes. D'apres les articles, le g doit-^etre
%compris entre 2 et 3, de maniere plus precise entre 2:1 et 2:4, pour les reseaux bio-
%logiques. Nous choisissons arbitrairement que notre reseau ideal aura un g de 2.2.
%Nous allons nous appuyer sur ces deux conditions pour implementer la fonction.
%
%Nous allons parcourir l'ensemble des noeuds du graphe et calculer la frequence
%pour chaque degre. Ensuite, nous comparons cette distribution a une loi de puissance
%en calculant la distance au carre entre la valeur reelle et la valeur theorique.
%-----------------------------------




\subsection{Propriété de petit monde}
% Anthony

\subsection{Formation de cliques}
% Anthony


%%%%%%%%%%%%%%%%%%%%%%%%%%%%%%%%%%%%%%%%%%%%%%%%%%%%%%%%%%
\section{Implémentation de la fonction de \textit{fitness}}
% Anthony


%%%%%%%%%%%%%%%%%%%%%%%%%%%%%%%%%%%%%%%%%%%
\section{Répartition des tâches et planing}
% Marion


%%%%%%%%%%%%%%%%%%%%
\section{Conclusion}

\end{document}